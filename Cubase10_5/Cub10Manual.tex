\title{Cheatsheet for Cubase 10.5 Pro}
\author{Kuratorium Monokromat}
\date{\today}

\documentclass[10pt]{article}

\pagestyle{headings}

\usepackage[margin=3cm]{geometry}
\usepackage[colorlinks=true, linkcolor=blue]{hyperref}

\begin{document}
\maketitle
\newpage

\tableofcontents
\newpage

\begin{abstract}
I sum up some useful functionalities of cubase which I don't have in my head all the time. I don't need to search through the full manuals everytime I need to know something.
\end{abstract}

\section{General Manual}\label{GeneralManual}

Progress: p. 1155/1202 Assembling Operations

\subsection{Skipped sections}

\begin{itemize}
	\item Surround Sound (p.685)
	\item Using MIDI Devices (p.798)
	\item Synchronization (p.1068)
	\item VST System Link (p.1080) This can be used for distributing CPU intensive stuff to other computers and servers
	\item Rewire (p.1111) For streaming audio between two computer applications (which in this case means between a Steinberg and a Propellerhead program. But Reaper seems also possible)
\end{itemize}

\subsection{My custom setup}\hypertarget{CustomSetup}{}

\begin{itemize}
	\item Edit $>$ Automation follows Events is checked
	\item Edit $>$ Preferences
	\begin{itemize}
		\item Editing:
		\begin{itemize}
			\item Set Default Track Time Type to Musical (this means they follow timing changes)
			\item Check Part Get Track Names
		\end{itemize}
		\item General $>$ Personalization: Author Stella, Company Kuratorium Monokromat
		\item MIDI: Check chase events for everything except SysEx
		\item Record: Check Stop after Automatic  Out
		\begin{itemize}
			\item Audio:
			\begin{itemize}
				\item Split Files When Recording Wave Files Larger than 4 GB
				\item Uncheck Create Audio Images during Record
			\end{itemize}
		\end{itemize}
		\item User Interface $>$ Track \& Mix Console Channel Colors: Check all boxes
		\item VST:
		\begin{itemize}
			\item Check Activate 'Link Panners for New Tracks
			\item $>$ Plug-ins: Check Suspend VST3 plug-in processing when no audio signals are received
		\end{itemize}
	\end{itemize}
	\item Project $>$ Project Setup
	\begin{itemize}
		\item Sample Rate: 96 kHz
		\item Bit Depth: 24 bit float (this is necessary to record 32 bit depth)
	\end{itemize}
	\item Studio $>$ Studio Setup $>$ VST Audio System:
	\begin{itemize}
		\item Processing precision: 64 bit
		\item Check Activate Multi Processing
		\item ASIO-Guard to high (for a more stable performance)
		\item Audio priority to normal
		\item Check Activate Steinberg Audio Power Scheme
		\item Check Adjust for Record Latency
	\end{itemize}
	\item Studio $>$ Audio Connections
	\begin{itemize}
		\item For line6
		\begin{itemize}
			\item Stereo Processed: Left $\rightarrow$ Send 1; Right $\rightarrow$ Send 2
			\item Stereo Dry: Left $\rightarrow$ Send 3; Right $\rightarrow$ Send 4
			\item Mono Processed: Mono $\rightarrow$ Send 1
			\item Mono Dry: Mono $\rightarrow$ Send 3
		\end{itemize}
		\item For Mackie
		\begin{itemize}
			\item Stereo: Left $\rightarrow$ Analog 1; Right $\rightarrow$ Analog 2
			\item Mono 1: Left $\rightarrow$ Analog 1
			\item Mono 2: Left $\rightarrow$ Analog 2
		\end{itemize}
	\end{itemize}
	\item Studio $>$ Mix Console
	\begin{itemize}
		\item Activate Channel Overview (or Meter Bridge), Channel racks, Control Room/Meter
		\item Right click on Level Meters $>$ Global Meter Settings $>$ Meter Peak Options: Activate \textit{Hold Peak} and \textit{Hold Forever}
	\end{itemize}
	\item Transport $>$ MIDI Cycle Record Mode: Stacked
\end{itemize}

If some plug-ins of any kind or extensions are missing, fix the path or checkbox in Studio $>$ More Options $>$ \textbf{System Component Information}

\subsection{Audio}\label{AudioQuality}

\subsubsection{Quality}

\begin{itemize}
	\item Set sample rate: Project $>$ Project Setup
	\item Set processing precision: Studio $>$ Studio Setup $>$ VST Audio System
	\begin{itemize}
		\item Consider that not all plugins support 64bit processing. Check this by accordingly setting the display options in Studio $>$ -in Manager
	\end{itemize}
	\item CAUTION: Recording needs RAM. 
\end{itemize}

\subsubsection{Useful Processing}

\begin{itemize}
	\item Audio $>$ Generate Harmony Voices can give thin tracks more substance. If the original track follows a chord track, the additional harmonies are distributed accordingly
	\item Project $>$ Convert Tracks $>$ Multi-Channel to Mono: Can split the selected (or all) stereo or surround track into several monos. Settings are kept. The other way round is possible in the same menu.
	\item Select Event, select Event Editor, open Hitpoints section, click Create Groove and you get an according \textbf{Quantize Preset}
	\item Use the draw tool to draw \textbf{Audio Envelopes} directly on the event. Remove the points by dragging them out of the event or selection Audio $>$ Remove Volume Curve
	\item Audio $>$ Process (those are all direct offline processes, which are applied to the clip itself, not the event)
	\begin{itemize}
		\item Normalize
	\end{itemize}
	\item To free some hard drive space, minimize the files. But be careful. This is permament. If you want a copy with minimized files, use the backup option. To minimize, select all the files in the pool and execute Media $>$ \textbf{Minimize File}.
\end{itemize}

\subsection{MIDI}

\subsubsection{Input}

On the top of MIDI track inspectors, there is a button for the input transformer, which allows you to filter out or change certain types of incoming MIDI messages before they are recorded - either locally or globally.

\begin{itemize}
	\item Manage your MIDI Devices in "Studio Setup $>$ MIDI Port Setup"
	\begin{itemize}
		\item You can specify here, which devices are included when selecting "All MIDI Inputs" on a track (Remote Control devices should be deactivated here)
		\item When experiencing persistent timing issues (like shift), activate "Use System Timestamp for 'Windows MIDI' Inputs"
	\end{itemize}
\end{itemize}

\subsubsection{Import}

\begin{itemize}
	\item Always activate Import Controller as Automation Tracks, so you can clearly see what automation there is
	\item Ignore Master Track Events on Merge kills tempo track information for the imported MIDI if checked
	\item Always Auto Dissolve Format 0. That way, embedded MIDI channels get their own track each
\end{itemize}

\subsubsection{Export}

In the export dialogue you have several options.

\begin{itemize}
	\item Export as Type 0: All MIDI data is on one track, but different channels. If this is not checked, all MIDI from separate tracks will be on separate tracks in the export, which is nice.
\end{itemize}

\subsubsection{Editors}

You can create continuous pitchbends with the line tool. When drawing a parabola, you can press ctrl (reverse), alt (position change) or shift (increase/decrease exponent). In sine/triangle/square the Length Quantize determines the event density and they can be influenced via ctrl (phase of beginning), alt-ctrl (position change), shift-ctrl (maxiimum position) or shift (set period).

The automation also contains a lane for articulations/dynamics, where you can set velocities in classical categories like mezzo forte and so on, which can be configured in the Dynamics Mapping.

The automation editor contains some nice smart controls with tooltips, when editing several events at the same time. For some, you need to push alt.

In order to move the automation with selected notes, either activate Auto Select Controllers in the toolbar or use Edit $>$ Select $>$ Select Controllers in Note Range

\paragraph{Key Editor}

The chord editing section to the left allows you to enter chords instead of single notes. It also detects known chords when they are selected in the note display and you can change them or their voicing (Inversions/Drop Notesa). The Create Chords Symbol function is very neat and creates the respective chords from your MIDI on the chord track.

\paragraph{Drum Editor}

\begin{itemize}
	\item Place multiple notes by just pulling the pencil accross. The notes are set according to the quantization. Snap should be activated for this. Deleting works just the same
	\item The "house" in the toolbar allows you to only display those drum sounds which contain actual events
	\item The button with the two rhombi allows you to display event length as in the key editor
	\item When using a drum machine that supports drum maps (like Groove Agent), you can import it to the instrument track: Go to Drum Maps pop up menu and select Create Drum Map from Instrument.
\end{itemize}

Drum Maps do input conversion and output conversion. Input conversion is for MIDI devices. Let's say the pitch is C, the I-Note is A and the O-Note is E. If you play an A on your keyboard, a C is recorded. If a C is played back, an E is sent to the vst instrument. But be careful. If you want to export the track or work on it with a different editor, you need to execute MIDI $>$ O-Note Conversion.

\paragraph{Score Editor} At first, use MIDI $>$ Scores $>$ Staff Settings (or double-click to the left of the staff) to determine staff mode (single or split), quantize (only applied graphically; activate auto quantize and Dev. to make it look as legibile as possible), clef/key, transpositions (e.g. by setting the correct instrument) and interpretation (e.g. syncopation, which prevents displaying a dotted note at the end of a bar with ties).

The signature follows the tempo track settings.

\paragraph{In-Place Editor}

Can be turned off and on with a specific button on the track. Could come in handy in some situations. Also nice: You can drag and drop stuff from one track to another.

\subsubsection{Expression Maps}

Can hold information about how notes are played (e.g. sustain, pizzicato, hammer-on, pull-off, strumming, palm mutes, sliding). Access via MIDI $>$ Expression Map Setup or in the Inspector of MIDI/Instrument tracks . The articulation is transmitted either by key changes or by program changes.

The maps can be extracted from VST3 instruments (e.g. track presets with pre-configured expression maps). If you have created an expression map, you should by any means save it.

Sound slots represent kinds of articulation. You can also create articulations that consist of several different articulations. In the remote area, define the key that triggers a certain slot from a MIDI device (latch mode means that the pressed key is active until another one is pressed - this is global for all expression maps). Articulations that are in the same group can not be used together.

The actual sound of the articulation needs to be defined in the Output Mapping.

The controller lane of MIDI editors provides a segment for articulations/dynamics, where different articulations are displayed in different rows. If they belong to the same group, they are shown in the same color.

Expression maps are also available in Score (Inspektor) and List (Comments) Editor.

\subsubsection{Note Expression}

Yield to the MPE standard. Each note gets its own MIDI channel, so expressions can be applied to individual notes. Thus you can apply expressions to single notes. This can also be seen in the editor. Not all plug-ins and devices support this.

Devices for this can be set up in Studio $>$ Studio Setup. Select your connected device there.

There are MPE presets of Padshop and Retrologue available: Go to the Media rack, then VST instruments, Padshop/Retrologue tile.

The inspector holds a tab for Note Expression settings. The list is divided in two: On top are VST note expressions, on the bottom are MIDI controllers. The expressions Input Assignment can be set via the L-button (learning). Push it and move the desired knob on your MIDI device.

Adding an expression to a note is as easy as double clicking that note and making the desired adjustments with the draw or line tool. You need to select the desired parameter in the left-hand inspector. Deleting note expressions is a bit clumsier (but there might be a shortcut): Double-click the note etc, draw a rectangle and entf. Copy/paste via ctrl+c/v. You can also assign a key command to Paste Note Expressions to paste without having to open the expression editor for that note. That would be especially handy for pasting to several notes (but you can also select them and double-click, so...). Copy-pasting also works inter-parameter.

There are also Smart Controls for the Notes.
\begin{itemize}
	\item Pull at the bottom right in order to increase the release length, so you can influence some ringing with note expressions.
	\item The other smart controls do the usual stuff
\end{itemize}

Some dedicated functions can be found in MIDI $>$ Note Expression $>$
\begin{itemize}
	\item Trim Note Expression to Note Length
	\item Remove Note Expression
	\item Note Expression MIDI Setup: Decide, which messages are used as note expressions. The Controller Catch Range lets you catch expressions that were sent slightly before the actual note. To record it, however, you need to activate MIDI as Note Expression in the Note Expression field inside the Inspector.
	\item Convert to Note Expression (is applied to MIDI control change messages)
	\item Consolidate Note Expression Overlaps: Necessary if the expressions of two notes overlap and the control changes might get confused.
	\item Distribute Notes to MIDI Channels: Necessary if not using a vst3 instrument
	\item Dissolve Note Expression: Converts Expressions to control changes.
\end{itemize}

\subsubsection{Useful functions from the MIDI menu} All commands are accessed by going to MIDI $>$ Command.

\paragraph{Transpose Setup} Scales can be applied and ranges can be defined for max-min values. It might be fun to do some experiments with this to find new interesting melodies.

\paragraph{Merge MIDI in Loop} This allows you for example to directly apply all the midi insert or send effects to an event and create a new event, which contains all this information.

\paragraph{Dissolve Part} Create several parts from one, depending on channel or pitch (nice for separating drum computer sounds).

\paragraph{Bounce MIDI} E.g. for bringing lanes together, just as with audio.

\paragraph{Functions $>$} 

\begin{itemize}
	\item \textbf{Legato}: Extends the MIDI events such that they reach the following tones. A gap can be specified in Preferences $>$ Editing $>$ MIDI
	\item \textbf{Fixed Length / Fixed Velocity}: Applies the length from Length Quantize / Note Insert Velocity to the selected notes
	\item \textbf{Pedal to Note Length}: If you have recorded something with pedal, the pedal values are applied to extend the respective notes. This enhances the options for editing.
	\item Delete Overlaps (mono/poly): Removes overlaps for the same pitch / different pitches
	\item Delete Doubles: Removes doub le notes on the same pitch and same position
	\item Delete (Continuous) Controllers: This might be, what I have been looking for
	\item Restrict Polyphony: For instruments that can't play arbitrarily many notes
	\item Thin Out Data: Thins controller data s.t. an external device does not have to process a huge load
	\item Extract MIDI Automation: This may also be what I have been looking for. It extracts the automation visible in the MIDI editor and puts it on automation tracks. Only works for continuous controllers
	\item Delete Notes: Can be conditioned on properties like minimum length and velocity
\end{itemize}

\paragraph{Logical Editor} Here you can write filters for MIDI notes, e.g. randomizing certain values like velocity or pitch. You can then assign them a key command. On p. 899 in the manual, there is a useful overview of the meanings of data 1 and data 2 for different event types. Or you can go to the dedicated \hyperlink{LogicalEditor}{section}.

\subsection{Special Tracks}

\subsubsection{Arranger Track}

In the editor, you can set up your specific order. The up/down buttons navigate to the previous/next part, the left/right buttons bring you to the first/last repetition in a loop. The \textbf{flatten} option takes the arrange chain and transforms the current project to a linear project according to the arranger track. There are several options, e.g. creating a new project from it. Clicking the arrow on the left of an arranger element puts the project cursor its start.

\subsubsection{Chord Track}

You can let other MIDI tracks follow the chords track.

The events have no end. The end is determined by the start of the next event. The events can consist of a root note (e.g. C), a type (eg maj), a Tension (eg 7) and a bass note (eg F) $\rightarrow Fmaj^7/C$.

Double-click an event in order to open the \textbf{Chord Editor}, where you can create new chords. Activate MIDI Input to play the desired chord on a device. Inside the editor, there is also a \textbf{Chord Assistant}\hypertarget{ChordAssistant}{}, which makes suggestions for the next chord (or for a chord in between to others, if you push the super non-intuitive Gap Mode button next to the complexity scale). As mode, you can set Common Notes in order to determine how many notes the chords should share.

In order to listen to the chords, you need to connect the track to an instrument track (done in the track controls).

Deactivate Automatic Scales in the Inspector to enable editing of Scale Events. Then click the first scale event and set root key and type in the info line. Now you can edit the other Scale events. There appears a keyboard when you double-click them.

If Adaptive Voicing is activated, only the voicing of the first chord can be changed. Voicings may become important when it comes to instrumental selection and genre. You can e.g. set it to piano or guitar. The configurable options (which determine the affected chords) depend on the chosen instrument.

There are some functions in Project $>$ Chord Track $>$
\begin{itemize}
	\item Chords to MIDI: You should prepare a track for this to work. Alternatively you can just drag and drop chord events to a MIDI or instrument track
	\item Map to Chord Track: Maps the selected events/parts to the chord track.Should also work with VariAudio stuff.
	\item Assign Voices to Notes: Depends on which voicing you chose. Is applied to the selected event(s).
	\item Create Chord Symbols: Extracts chords from the selected MIDI events or notes and adds them to the chord track
\end{itemize}

Crazy stuff: Create an instrument or MIDI track, choose \textbf{Chorder} as MIDI insert effect. Drag and drop chord events. Magic.

More crazy stuff: You can drag/drop chord events to HALion Sonic SE pads - all subsequent chords are automatically mapped to the following pads.

Instrument or MIDI track inspector $>$ Chords $>$ Live Transform $>$ Scales/Chords: Makes sure that every key you hit is mapped to the scale/chord. Use Follow Chord Track to make an existing recording follow a chord progression.

You can record directly to the chord track. But also record enable a separate instrument or MIDI track.

\subsubsection{Input Tracks}

They can be seen in the mixer. You can also put effects here, but be cautious: The recorded audio file can't be stripped of those effects afterwards. It might, however, be good, to put some gate or slight compression here, so you can reduce the CPU load if you have to mix huge songs. When using such effects, always record in 64 bit depth to avoid clipping.

\subsubsection{Multitrack}

Some insert effects are not able to be applied to more than one (mono) or two (stereo) channels. By default, they are applied to the first sub-channels. You can influence this behaviour in the Channel Settings Editor $>$ Inserts section $>$ Routing tab.

\subsubsection{Sampler Track}

Can be handled like MIDI, but plays back the sample you have loaded in \hyperlink{SampContr}{Sampler Control}. Pay attention to the set root key.

\subsubsection{Signature}

You can also assign click patterns to signature changes, in order to get a specific groove (double-click the small plus at a signature event).

\subsubsection{Tempo}

The events have nice smart controls, which are pretty similar to those of other events.

Right-clicking the track lets you select all the events.

If your track has a fixed tempo, but you don't know it yet, you can use Project $>$ Beat Calculator and use the Tap Tempo option, to tap on your space bar.

Material for Project $>$ Tempo Detection must be at least 7 seconds long.

\subsubsection{Transpose}

With this you can transpose the whole song (and incoming recording) by several notes. It refers to the \textbf{Project Root Key}, which you can set in the Project window's toolbar. The Transpose Track's context menu gives you the option to \textbf{set root key for unassigned events}. But always make sure to always exclude MIDI percussion and FX (select the parts and set Global Transpose to Independent) when setting a root key or doing transposition.

The track has a button with an up- and down arrow inside brackets. This makes sure, transposes stay inside an octave, so you sounds don't get unnatural high or low.

When recording, existing transposition and root key is not taken into account.

The key editor toolbar holds an Indicate Transitions button. If pressed, the notes are shifted to the value of their true sound after being transpositioned.

You can change transposition also for events in the info line.

\subsubsection{VCA Faders}

They serve as remote controls for grouops of channel faders. They belong to a link group and control volume, solo/mute, listen, monitor and record. It can control only one link group. The meter displays the summed level of all channels linked in the group.

\subsubsection{Video}

Very limited. The only export format is Full HD. Supported Containers are MOV, MPEG-4 and AVI, but not all usual Codecs and framerates.

In Studio $>$ Studio Setup, you can set up Output Devices for Video, e.g. if you have some hardware from Blackmagic Design.



\subsection{Sampler Control}\hypertarget{SampContr}{}

You can transoform the sounds and even export them (to Groove Agent (SE), HALion, Padshop Pro/2 and maybe other VSTs).

Simply drag some audio or MIDI in (from MediaBay, Project window or file explorer). Make sure to set your loop mode correctly. The harry potter scar hard resets everything, if you have a sample playing endlessly because you made dumb decissions.

You can do all the funny stuff like reverse the sample, only replay it at a fixed pitch, may it only replayable monophonic and transfer it to a new instrument (on a dedicated track).

In addition to set the sample start and end with smart controls, you can also set a sustain loop (depending on which loop option you have decided). Both with fades.

With AudioWarp you can adjust time stretching and formants (solo voice). If AudioWarp is deactivated, the sample is replied faster for higher pitches and vice versa. If you activate Legato, the Sample won't start from the beginning when a different note is hit, but just continue where it is now.

In the Pitch section, you are able to set a glide value, which is applied between notes.

Then you can still manipulate the sample by filtering it, applying amps and envelopes (small button next to Amp). The envelope has some modes:

\begin{itemize}
	\item Sustain: Envelope is played from the first node to the sustain node, then held as long as the note is played.
	\item Loop: Applies the envelope from the first node to the loop nodes and repeats the loop. Thus you can add motion to the samples sustain time.
	\item One Shot: Plays from the first to the last node, even if the key is released
	\item Sample Loop: Preserves the sample's natural attack. Decay is not applied before the sample has reached the start of the loop
\end{itemize}

On the keyboard, you can set the root node, specify the pitch bend range as well as the range of the sample

\subsection{Chord Pads}

Can be found next to the Editors in the lower project zone. Pay attention to the button on the upper left. If it is activated, the chords are sent to all tracks, that are monitored or record-enabled. If it is deactivated, the chords are only sent to the tracks that have Chord Pads in the input routing.

The blue keys are the ones that trigger the pads, when played on a MIDI keyboard; the green keys can change voicings, tensions and transpose settings of the pads (this can be set on the Pad Remote Control page of the Chord Pads Setup dialog in the bottom left); the brown keys trigger sections (Player Mode needs to be set to Sections; you can change the keys in the Chord Pads Setup dialog).

The Chord Pads setup also allows you to change the pad appearance, number of octaves etc.

The pads have smart controls to
\begin{itemize}
	\item Open the chord editor (mid-left)
	\item Blue border: chord is used for suggestions in the Chord Assistent; Yellow border: chord pad is set to adaptive voicing, meaning that all other chord pads follow its voicing. If no pad has a coloured border, adaptive voicing is applied automatically (except if you lock them, see upper right AV vs L)
	\item The arrows on the right allow you to change the voicing
	\item The arrows on the bottom allow you to add/remove tensions
\end{itemize}

In the context menu, you can set a pad to origin for the chord assistant, assign pads from MIDI input and do some other stuff. You can also assign the pads from chords on the chord track, when using the functions menu (downward arrow to the left). The functions also allow you to tie the pads to the grid, such that the playback of a triggered pad starts at the next defined musical position. This is especially worthwhile when working with arpeggiators. It is also possible to transpose all pads.

The comment-like button to the left opens the \hyperlink{ChordAssistant}{Chord Assistant}.

The e-button brings you to the Player Setup, which is full of useful stuff. For example you can choose Pattern as Player Mode to play arpeggios. A MIDI loop or part can be used as pattern, e.g. via drag\&drop, but the loop mst have between 3 and 5 voices - you can see the voices in the \hyperlink{MediaBay}{Media Bay}. The Player Setup can be different for each track.

Drag-dropping chords swaps them. Alt-drag-dropping copies the chords.

\subsection{Events}

\subsubsection{Audio (and alignment)}

Events are not the same as files (which are referred as audio clips). The event triggers the playback of one or several files.

\begin{itemize}
	\item Audio $>$ Bounce Selection creates a new file from an event
	\item Audio $>$ Open Audio Alignment Panel (or the button in the Project window toolbar)
	\begin{itemize}
		\item Match words: For vocals. It matches the lyrics. Very wow.
		\item Audio in musical mode must be bounced first
		\item Does not work on tracks modified by VariAudio or AudioWarp
		\item The used algorithm can be changed in the Sample Editor toolbar (e.g. élastique)
	\end{itemize}
\end{itemize}

\paragraph{Sample Editor}

Can edit several Events at the same time.

Can apply time stretch.

Can directly edit Audio Samples with the draw tool (e.g. for removing audio clicks by hand)

You can select a range and drag it to an audio track in the Project window to create a new Event.

A sample can be divided into regions. This might be useful for longer takes or podcasts. You can also create regions from hitpoints.

The super usefull menu on the left holds:
\begin{itemize}
	\item Definition
	\begin{itemize}
		\item Displays a second ruler with the sample bars. If it is different from the project's, the musical mode will adjust the audio sample according to the selected algorithm.
		\item Auto adjust exteracts a tempo definition grid for the audio sample. If it doesn't fit, the manual edit has some powerful manipulation options
	\end{itemize}
	\item AudioWarp
	\begin{itemize}
		\item Activating the Musical Mode automatically fits the audio file to the beats and bars of the project. The audio events behaves like a MIDI event.
		\item Can add swing (musical mode is necessary) according to the grid resolution
		\item Free warp allows you to just drag around the audio and time stretching is applied automatically. Dragging the warp markers at the top triangle allows you to correct their position. Alt+clicking deletes the markers (can also alt-click-draw a rectangle).
		\item If you have fucked up, go to Audio $>$ Realtime Processing $>$ Unstretch Audio
	\end{itemize}
	\item VariAudio
	\begin{itemize}
		\item Any Offline Processing or plug-ins should be applied \textbf{before activating VariAudio}
		\item Optimized for monophonic recordings
		\item Select Relative Pitch Snap Mode if the tones should keep their off when moved to a new note
		\item Select a MIDI reference track at the bottom. This will be usefull for vocal lines
		\item Formants are basically the vocal overtones. Shifting this should not affect the pitch or timing
		\item Each segment has smart controls:
		\begin{itemize}
			\item Top left/right rectangles: tilt the pitch curve. Alt+drag rotates the curve
			\item Top center rhomb: Tilt/Rotate Anchor
			\item Top center rectangle: Straighten the pitch curve
			\item Top left/right triangles: Range for straightening the pitch curve (default: entire pitch curve)
			\item Left and right rectangle: Warp start and end. Can be corrected with alt+drag
			\item Bottom right rectangle: Volume
			\item Dotted bottom line: Glue segment to adjacent sections, if they should belong together
			\item Straight bottom line: Split segment
			\item Bottom rectangle: Quantize the pitch to tthe nearest semitone
			\item Bottom left rectangle: Shift segment formant
			\item The segments' color can be adjusted in the toolbar
		\end{itemize}
		\item Shift + Double-click selects all following segments of the same pitch
		\item Alt-click on MIDI-Input to activate step mode: After assigning one segment, the next one is automatically selected
		\item When VariAudio is open, you can range select notes with the same tone Edit $>$ Select.
		\item The functions allow you to extract MIDI from the audio. Thus you can underlay it with some nice pads
	\end{itemize}
	\item Hitpoints
	\begin{itemize}
		\item They are generated automatically, but can be edited by fine-tuning the detection parameters or by hand (shift-click to remove, alt-click to add, drag-drop)
		\item There are some processing options like 
		\begin{itemize}
			\item Produce slices, Markers, regions or events (for multiple tracks like drums use Audio $>$ Hitpoints $>$ Divide Audio Events at Hitpoints together with group editing)
			\item Create a Groove Quantize Map: The Groove is applied to the Quantize Panel in the toolbar
			\item Create Warp Markers: They can then be edited in the AudioWarp section (see above)
			\item Create MIDI Notes (with a fixed pitch and length)
		\end{itemize}
	\end{itemize}
	\item Range
	\begin{itemize}
		\item Allows you to create a Sampler Track from a selected range.
	\end{itemize}
	\item Process
	\begin{itemize}
		\item You can create a region from a selected range or event
		\item Direct Offline Processing can be applied to a range or event as well as insert effects or other audio processing
		\item If all this processing consumes too much \textbf{CPU}, you can use thee respective option in the menu or (for multiple files) select the affected audio events and Audio $>$ Realtime processing $>$ Flatten Realtime Processing
	\end{itemize}
\end{itemize}

\subsection{Effects}

\subsubsection{Offline Processing}

Open via Audio $>$ Direct Offline Processing or F7.

Can help to largely reduce the used disc space. When performing offline processing, the original files are kept. But if you edit stuff to prepare it for mixing, you could also throw away the old files when you are done and happy with the result. Another advantage is, that it reduces CPU usage.

The tool itself is very flexible. You can add plug-ins, fx chain presets, track presets even by dragging them from inserts. You can even drag entire racks from the mixconsole into the DOP Window or load FX chain presets and track presets (Select Preset). And you can copy/paste between several processed Audio files (context menu in the processing list).

With the audition button in the toolbar, you can play the part with the effects from the first to the selected one. You can also add tails and adjust the range, in case you use delay or reverb or smth.

DOP can be applied to (multiple) clips in the Project Window, in the Pool and to ranges in the Sample Editor.

For effects with learning effect (e.g. noise reduction) you should deactivate Auto Apply and use the \textbf{Audition Loop} and activate Audition. Train it on a piece that only contains noise, then click discard (the plug-in keeps the parameter settings), select the whole event(s) and apply.

If a stereo effect is applied to a mono channel, the left side of the effect's output is used.

If you start to use this stuff frequently, you can set up your favorites. If you drag multiple effects in this section, they are batched, so you can apply them more easily. But be careful: batches are always instantly applied, even when Auto Apply is deactivated.

If everything is nice: Audio $>$ Make Direct Offline Processing Permanent or choose Make All Permanent in the DOP Window.

Built-in effects are: Envelope (for volume), fades, gain, phase invert, normalize (peak or loudness) [for track : peak $\in [-50, 0] dB$, loudness $\in [-34, 0] LUFS$], pitch shift (also working with chords and with or without time stretch), envelope-based pitch-shift (the curve determines the height), remove DC offset option (strom; determine if there is an offset with Audio $>$ ), Resample, Reverse, Silence, Stereo Flip and Time Stretch

\paragraph{Time Stretch and Pitch Shift algorithms}

\begin{itemize}
	\item élastique
	\begin{itemize}
		\item Poly- and monophonic
		\item Pro Formant includes formant preserevation (but what is formant?)
		\item Time mode favors timing accuracy, Pitch mode favors pitch accuracy
	\end{itemize}
	\item MPEX
	\begin{itemize}
		\item Has a solo (monophonic) and poly (mono- and polyphonic) mode
		\item Musical mode recommended
		\item Poly Complex has high CPU consumption
	\end{itemize}
	\item Standard
	\begin{itemize}
		\item Optimized for CPU efficient real time processing
		\item Holds presets for Drums, Plucked, Pads, Vocals, Mix and Solo (monophonics)
		\item Warp settings can  be customized. Low grain size for material with many transients, higher Overlap for audio with stable sound character
	\end{itemize}
	\item All of the algorithms can lead to degradation in audio quality and produce artifacts.
\end{itemize}

\subsubsection{Insert Effects}

Post-fader effects are also post-EQ. This is good for things like Dithering or maximizing.

In order to reduce processor power, you can freeze them. But the tracks can not be edited anymore. But make sure to add some tail, so reverb and stuff aren't cut off.f

\subsubsection{Send Effects}

\begin{itemize}
	\item By right-clicking, you can make them pre- (blue) or post-fader (orange)
	\item In the channel editor, you can activate \textbf{Link Panners} from the functions (small arrow in the top right)
\end{itemize}

\subsubsection{Side-Chain Input}

Available for modulation, delay and filter.

Especially good for adding compression to a bass when the drums are hit. Or for ducking a delay when the track makes a noise. For the latter, duplicate the track and set the duplicate as Side-Chain Input for the delay. When you use the side-chain of a modulation, the input's envelope is used for modulation, which can do crazy stuff.

\subsubsection{Extensions}

Those are handled different than plug-ins. E.g. they can have their own editor. Melodyne is an extension. You can't perform Hitpoint, Warpstuff, DOP, Time Stretch, Harmonies or Audio Alignement with Events that have an active extension. To do that, you have to bounce the event.

\subsection{Automation Panel}

Open with F6. Useful for displaying/suspending/... reading or writing automation of a certain type. Also there are some options which influence, how writing is applied.

There are some modes for writing automation:
\begin{itemize}
	\item Touch: Punches out as soon as the automation control is released and returns to previously set value (return time is adjustable).
	\item Auto-Latch: When releasing the controller, the last value is held until punch out
	\item Cross-Over: Allows smooth transitions between new and existing automation. As soon as you cross the old automation curve, the writing punches out.
	\item Trim: Allows you to adjust already written automation
\end{itemize}

Fill options (when Touch is activated) - they work with the drawing tool, too:
\begin{itemize}
	\item To Punch: Move the fader until you have the value you want. On release, you punch out and the value is set for the time between punch in and punch out
	\item To Start: Same as the punch, but between Project start and punch out
	\item To End: Between Punch out and Project end
	\item Loop: Punch out value is set between the Locators
\end{itemize}

In the settings tab, you can - amongst other options - check Continue Writing on Transport Jump. This will help you writing automation when jumping here and there or when the replay passes a cycle or if you are using the arranger functions. The Reduction Level value influences, how much an automation curve is smoothed after writing.

When using an external MIDI device with automation capabilities, you can set the behaviour in MIDI $>$ CC Automation Setup.

SysEx contains information for external MIDI devices.

\subsection{Control Room}

You can set up your whole control room setup (including talkback, cue, headphone, additional inputs, multiple monitors, etc) here. This might be cool as soon as you have an external cubase controller.

The channels are set up in the Audio Connections. Ports can not be assigned to a Control Room Channel and a regular Bus or Channel at the same time.

Metronome can be activated separately for cues.

In Channels, you can solo individual channels to make sure they are wired correctly.

You can define Downmix Presets, if you want to do more than one mixdown at a time.

The main output has a volume knob which does not affect the actual Main Mix level for mixdowns.

You can select channels in the mix and then create a cue from them. Use the context menu in the cue area.

\paragraph{Inserts tab.} Here you can adjust things like input gain and phase, add insert effects (brickwall limiter would be good to avoid accidental overloads). All Monitor Inserts are past Control Room Fader.

\subsection{Media Bay}\hypertarget{MediaBay}{}

Some of the advantages are also mentioned in the section about the \hyperlink{MediaTab}{Media Tab} amongst the efficient workflows.

The checkboxes in the Filebrowser tab to the left indicate, if the folder is scanned. Orange checkmark indicates, that at least one subfolder is not included in the scan. Red folders are currently being scanned, whites have bin scanned, yellows have at least one unscanned subfolder.

The preview section at the bottom can align the file to the project beat. This may apply real-time time stretching

The search function can handle boolean operators:
\begin{itemize}
	\item and: and / + (default operator $\rightarrow$ always set, if no other is specified)
	\item or: or / ,
	\item not: not / -
	\item parentheses: ()
	\item quotation marks: for sequences of words, to evade the default and-operator. Or words with hyphens, to evade the not-operator
	\item wildcards: *
\end{itemize}

If you choose mulitple attributes in the Logical Search, they are by default connected with an xor-operator.

The context menu in attribute search enables you to find files which have attributes in common.

The attribute editor enables adding attributes to files. They can act as kind of labels, which make navigation easier for you.

In order to make the attributes, you set in the media bay, accessible on other systems, you will need to create a \textbf{Volume Database} by right-clicking an external medium in the filebrowser. Available Volume Databases are mounted automatically.

Samples can be previewed from the media bay and their beats can be transformed to the project tempo by activating a button at the bottom. If the sample is then dragged into the project, the name also contains the tempo (and the key, if there is one).

\subsection{Mixing and Mastering}

\begin{itemize}
	\item You can normalize your whole track: Audio $>$ Processes $>$ Normalize
	\item Display Peak Level, DC offset, loudness (RMS) and much more with Audio $>$ Statistics
	\item Have look at the \textbf{Spectrum} of a part (or song, during mastering) with Audio $>$ Spectrum Analyzer
\end{itemize}

\subsubsection{Mix Consoles}

\paragraph{Mix Console in Project Window}

In the left portion, you can access the pages "Faders", "Insert Effects" and "Send Effect".

\paragraph{Mix Console (F3)}

\begin{itemize}
	\item You can take snapshots of your current settings (Left side) to compare different versions of a mix. Up to 10 snapshots can be saved. Automation data is excluded from this. Youo can also take useful notes below the snapshot section.
	\item Have a look at the visibility stuff written somewhere above
	\item Toolbar:
	\begin{itemize}
		\item Rack can be used to show everything contained therein
		\item The state buttons enable you to bypass inserts, sends, eq, ...
		\item Super powerful link group, like suspend linking ()sus), absolute mode (abs) and temporary link (Q-Link $\rightarrow$ shift + alt + click is super convenient for configuring several tracks at once without linking them permanently)
		\item Configuration (in addition to the configuration in the left zone) also holds zone settings and show/hide of racks
		\item Linking channels (can be configuered pretty freely):
		\begin{itemize}
			\item Adds a \textit{display line} for all channels, where you can change the settings and add and remove channles from the group
			\item Make changes on a single channel by alt + click it or activate \textit{Sub} in the toolbar
			\item Make absolute instead of relative changes by activating \textit{Abs} in the toolbar
			\item You can assign a VCA Fader
			\item Automation tracks are not affected
		\end{itemize}
		\item In the funktions, you can \textit{Reset MixConsole Channels} to bring volume faders to 0 dB and pan to the center. You can also link several mix consoles so you can display the same stuff on different monitors
	\end{itemize}
	\item Move faders slowly by pressing shift
	\item Stereo in/stereo out channels can have a special panner type (context menu)
	\item Exclusive solo via strg + click solo, solo defeat via long click or alt + click
	\item With the listen button (below solo/mute) you can send the signal of a channel directly to the control room
	\item Drag and drop one channel onto another to copy its settings (press alt if you also want to copy directt and output routing)
	\item The rack is super convenient for routing
	\item \textbf{Low- and highcut} should be done in the pre-section (this is still displayed in the EQ settings, which is nice), just as \textbf{phase and (pre-)gain} stuff
	\item Inserts:
	\begin{itemize}
		\item In the context menu click \textit{Set as last Pre-Fader Slot} to enable \textbf{Post-Fader Slots}
		\item In the context menu of some effects, you can activate side-chain
	\end{itemize}
	\item In the graphic EQ, modify \textit{only frequency} by holding alt, \textit{only gain} by holding ctrl, \textit{only Q} by holding shift.
	\item The channel strip holds some built-in processing, so you don't need to put that necessarily into the inserts. Those elements can be moved by drag and drop.
	\begin{itemize}
		\item Noise gate with thresh, range, attack, release, monitoring option, internal side-chain (frequency, Q factor)
		\item Compressor: Standard, Tube (smooth and warm; internal side-chain) or Vintage (has Attack Mode which conserves the punches)
		\item EQ with 4 bands
		\item Tools:
		\begin{itemize}
			\item DeEsser with Solo function for the frequency band that is searched for sibilants and a diff function, so you can hear the signal that is removed from the original
			\item EnvelopeShaper for boosting or attenuating attack and release phases
		\end{itemize}
		\item Sat (adds warmth)
		\begin{itemize}
			\item Magneto II and Tape Saturation: Simulate analog tape machines. HF-Adjust sets the amount of high frequency content of the saturated signal. Has a solo function so you can determine the appropriate frequency range
			\item Tube Saturation: Siomulates compression of analogue tube compressors
		\end{itemize}
		\item Limit: Brickwall Limiter, Maximizer or Standard Limiter. Brickwall creates a latency of 1 ms, so only use it for mixdowns.
	\end{itemize}
	\item Send: If you don't have the desired fx channel yet, you can click on an send-slot and then select \textit{Add FX channel...}
	\item Cue is rooted to he Control room. They are kind of aux sends. Cue channels need to be added to the Control Room in the Audio Connections (F4)
	\item Direct routing is post-fader and -pan and allows to create \textbf{different mix versions in one go}. Add the same destination for multiple tracks by selecting them and then shift-alt-clicking the first slot. Switching destinations can be automated by activating write for the desired channels and then switching during playback. Here again you can hold shift-alt to switch all selected channels. The context menu allows you to activate summing mode $\rightarrow$ you can select \textbf{several destinations}. But you may rather use sends for such things. The first option in the direct routing should always be the one with the most subchannels.
	\item Quick controls allow you to access things you tend to change often quickly
	\item Device Panels is for external MIDI devices, audio track panels or VST insert effect panels
	\item The latency display shows the per-effect-latency in detail
	\item Make use of the Save as Default Preset option for stuff you always adjust, e.g. disable auto make-up in compressors
\end{itemize}

\subsubsection{Channel Settings (e)}

You can see EVERYTHING here. Everything you can manipulate in a channel.

\begin{itemize}
	\item Super mighty navigation. Left-right goes to previous edited channel, up-down navigates through the mix console. The left-right in the middle navigates through the audio chain, i.e. to in- and output channels
	\item You can display the \textbf{Output Chain}
	\item The EQ has a comparison mode. Super nice. There are also a few options concerning the display referring to FFT if you click the gear.
\end{itemize}

\subsubsection{Metering}

In the right of the project window. Allows you to estimate what to changes to make on your whole mix.

EBU (European Broadcasting Union) scale is the recommended standard. The scales reach from from -18 LU to 9 LU / - 31 LUFS to 14 LUFS (EBU +9) or from -36 LU to 18 LU / -59 LUFS to -5 LUFS (EBU +18).

LU are measured in dB, as well as LUFS. the latter can be seen in the AES17 scaling.

As for loudness (second tab at the bottom of Meter), there are those thingies:

\begin{itemize}
	\item LUFS (Loudness Unit, referenced to Full Scale) is the average loudness measured over the whole track. Audio should be normalized at -23 LUFS ($\pm 1 LU$) - also called \textit{Integrated Loudness}. That is displayed by the triangle to the left of the meter scale
	\item \textit{Short-Term Loudness}: Measured every second on a block of 3 seconds, to give information about the loudest passages. This is displayed by the triangle to the right of the meter scale.
	\item \textit{Momentary Loudness}: Measured every 100 ms in range of 400 ms
	\item LU (Loudness Units): Dynamic range over the whole title - ratio between loudest and quietest non-silent sections. This is displayed as range in the Loudness tab. Can help deciding, how much \textbf{compression or expansion} the track needs. A recommendation for highly dynamic music (e.g. for movies) is a 20 LU.
	\item \textit{True Peaks} can be measured in opposition to digital peaks. This helps avoiding clipping and distortion. Maximum should be -1 dB.
	\item In the settings, you can define when clipping should be indicated for Momentary Max, Short-Term, Integrated and True Peak. The AES17 standard adds an offset of 3 dB to the RMS value.
\end{itemize}

\subsection{Audio Export}

\begin{itemize}
	\item For CDs, use 44.100 kHz and 16 bit. Also add dithering (\hyperlink{UV-22HR}{UV-22HR}), if you recorded in better quality.
	\item Use realtime export if you have external effects or instruments; or plug-ins that require time to update correctly during the mixdown.
\end{itemize}

\subsection{Exporting and Archiving Projects}

\begin{itemize}
	\item Make sure the project is self contained:
	\begin{itemize}
		\item Media $>$ Prepare Archive verifies that all referenced clips are contained in the project folder (does not account for videos)
		\item File $>$ Back up Project: Stores the project in a new location. Original project remains unchanged.
		\item The Update Display option makes sure, that you can see, if and where clipping (or general high peaks etc.) occur.
	\end{itemize}
\end{itemize}

\subsection{Tips for efficient workflow}

\begin{itemize}
	\item In the \textbf{Left Zone}, there is a visibility-tab which allows you to hide tracks. You can create custom configurations for this with the respective tool from the toolbar
	\begin{itemize}
		\item Click the equals sign next to "Visibility" in order to syncronize the tracks' visibility with exactly one mix console visibility (Channels in the left/right zone of the console as well as tracks in the top part of a divided track list are not affected by this)
		\item The settings are also applied to the Mix Console in the Lower Zone
		\item When the visibility tab is open, you can click the Zone-Tab in the bottom. This allows you to lock channels to the right/left of the mix console
	\end{itemize}
	\item Right click on a track and go to \textbf{Track Controls Settings} (or click the gear at the bottom of the track list) the ) to adjust the available controls and their layout for each track type
	\item On top of the track list, there is a useful visibility filter (Press return when several tracks are selected to toggle them all)
	\item You can activate the \textbf{Transport Bar} at the bottom of the Project window in the "Set up Window Layout" menu
	\item Opening multiple tabs in the Track/Editor Inspector: strg + click
	\item The \hypertarget{MediaTap}{\textbf{Media} tab} in the \textbf{Right Zone} contains basically everything from VST-Instruments to samples and a shitload of customizable presets. There even is a file browser, yay! You can drag and drop vst effects to either insert or send them somewhere or create an FX channel track. Same goes for Track Presets. For more, you can open the \hyperlink{MediaBay}{Media Bay} wiht F5.
	\item On the top right of the Project Zone, there is a slider which lets you zoom in on wave parts
	\item Edit $\rightarrow$ History allows you to navigate really fast and smooth by moving the separator, allowing you to quickly compare changes
	\item Editing
	\begin{itemize}
		\item alt + shift + 1: Combinates object selection (lower half of track) with range selection (upper half of track) (also works in track automations $\rightarrow$ you can automate sections	)
		\item Pressing 1 repeatedly: Toggling stretching modes: Normal, sizing moves contents, sizing applies time stretch
		\item strg + alt in object selection mode: moves the contents within a a part
		\item strg + shift in range selection mode: creates global range
		\item When having recorded several takes, you can show them by clicking the Show Lanes button for the track
	\end{itemize}
	\item When adding new tracks, don't righ click, but use the '+'-symbol on the top left of the track list. So you can immediately set configurational stuff.
	\item Work with track versions instead of different cubase files
	\item Things for which macros might be usefull:
	\begin{itemize}
		\item Put selected tracks into folder and add a group channel
		\item Duplicate track without data
		\item Export whole song
		\item Move selection to new track version
	\end{itemize}
	\item File $>$ Import $>$ Track from Project: Allows you to import a certain track from another Cubase project
	\item For tracks inside \textbf{Folders}, you can activate \textbf{group editing}. Every edit done to a track in that folders is also applied to the other ones.
	\item Project $>$ Track Folding $>$ Move Selected Tracks into New Folder
	\item When changing \textbf{signature} or \textbf{tempo} in a song, just use the goddamn signature and tempo tracks
	\item In the signature track, you can also set up \textbf{click patterns}. Even triplets in 4/4 signature is possible. Click pattern can also be set in the transport bar.
	\item Selecting something and pressing \textbf{ctrl + d} can \textbf{duplicate} almost anything
	\item You can save the settings on a track as a preset (context menu) and then create tracks from the preset (the sign next to the plus over the tracks). They can be categorized with attributes in the MediaBay. Presets for several tracks are also possible. But in that case you might just want to create a template. Also nice: Most components (like insert effects, EQ, ...) have a From Preset option, when right-clicking. This extracts the specific portion from an existing preset.
	\item Changing height of all tracks: Hold ctrl while dragging
	\item Use the \textbf{Draw tool} to super nicely adjust stuff like \textbf{Clip volume process}
	\item Press ctrl + return when renaming a track $\rightarrow$ \textbf{renames all the events}
	\item Selecte the cut-tool and press alt while cutting $\rightarrow$ The event is \textbf{split into parts of equal length}
	\item Holding shift while repeating an event creates a shared event (MIDI). It can later be converted to a real copy (Edit $>$ Functions $>$ Convert to Real Copy)
	\item Select parts, that belong together (e.g. in a part) and then Edeit $>$ Group. Selecting, moving, duplication, resizing, fades, splitting, locking, muting and deleting is applied to the group as a whole. Also every folder track has a group editing symbol (=) which allows you to handle the events inside the folder as a group.
	\item Phase of an audio event can be inverted by selecting the Invert Phase option in the info line
	\item Edit $>$ Select $>$ Equal Pitch can select all the notes with the same pitch in an event. You can also shift-double-click a note to do this.
	\item The click pattern can be displayed in the transport bar
	\item If you have no MIDI keyboard at hand, try Studio $>$ \textbf{On-Screen Keyboard}
	\item In Studio $>$ Studio Setup $>$ MIDI Port Setup, you can name the midi devices (display as) so you don't have to gues which one you need to use as input
	\item When recording MIDI, you can record the notes and other stuff (like pitchbend, modulation, sustain, volume) on separate tracks, but replay them at the same time by sending them to the same output and MIDI channel. Same goes for program changes.
	\item MIDI $>$ Reset if something like pitchbend or vibrato or sustain is hanging
	\item Quickly quantizing MIDI notes by selecting the desired value and pressing q. Note length is maintained. They can be quantized by Edit $>$ Advanced Quantize $>$ Quantize MIDI Event Lengths (this cuts off the ends, so the lengths fit the defined value)
	\item The same can be done for audio events, if AudioWarp Quantize is activated. Quantizing aligns the warp maarkers with the quantize grid and stretches or compresses the stuff in between. Also works on several audio tracks with same start and end time in a folder which has Group Editing activated
	\item When setting up a quantization, use Catch Range (for MIDI, only parts within the range are considered in quantizing; for Audio, hitpoints in each others range are considered to belong to the same beat) and Non-Q (events within that zone are not quantized, so you keep the groove alive). If there are no slight variations, you can create them with the Randomize value. Activate Mode, if you don't want the events to fully move to the specified grid positions, but only up to a specified percentage. Activating MIDI CC also moves the control data. Pre-Q, if available, lets you quantize to a grid first, before it's quantized to your groove.
	\item You can capture the groove of MIDI and audio events by simply dropping them into the Quantize Panel. Awesome!
	\item Punch in/out: If you deactivate the lock, you can set the punches between the locators, which is usefull when the performer needs much intro
	\item When you have selected several Tracks, you cann select \textbf{Add Group/FX/VCA Channel to Selected Channels} from the context menu in the mixer or Add Channel $>$ Group/FX/VCA to Selected Channels from the context menu in the track list
	\item If you need to rehearse a (partial) mixdown inside the Project, you can use Audio $>$ \textbf{Bounce Selection}. Thus the mixdown is automatically added to the pool.
	\item In order to manage recordings from separate sessions, you can create separate record folders via Media $>$ Set Pool Record Folder. Then it is clear, what was recorded in which session.
	\item Use libraries for snippets, loops, clips etc. that you use in more than one Project for easy access. To create one, go to File $>$ New \textbf{Library}. Here, you can also save and open libs.
	\item Super nice short-cuts:
	\begin{itemize}
		\item F6: automation-panel
		\item Q: Automatically adjusts the selected midi part
		\item J: De-/activate snap
		\item L: Locator jumps to the beginning of the selected part
	\end{itemize}
	\item alt-click to draw several automation events. Also the line tool holds several nice shapes you can apply instead of drawing them like an idiot
	\item Get the tempo from a recorded audio or MIDI via Project $>$ \textbf{Tempo Detection}
	\item Edit $>$ Render in Place $>$ Render Settings may help you improve your CPU load.
	\item Alt+K: onscreen keyboard
	\item Ctrl+Shift+D/N/H/G: Duplicate/New/Next/Previous Track Version
\end{itemize}

\paragraph{Key Commands \& Macros} Edit $>$ Key Commands. Everything else is pretty straightforward.

Assigning a command to a function that already has one, does not replace the previous command.

Enable Show Macros to assign commands or create new ones. The macros are also available at Edit $>$ Macros.

It is possible to export and import key command settings, which is useful, if you switch your workstation.

\paragraph{Remote Controlling Cubase.} You can connect a MIDI device via MIDI or usb. In order to not influence ongoing MIDI recordings, you should remove the controller from All MIDI: Studio $>$ Studio Setup $>$ MIDI Port Setup and deactivate "In 'All MIDI Inputs'". Here you can also find compatible devices when clicking the + to add a new device. If it isn't supported, simply use the generic and you can configure it yourself. The Relative flag needs to be set, if the controller is infinte (like some wheel). There also is a learn mode. You can set key commands for remote controllers.

\paragraph{Workspaces \& Profiles} Saves window layouts and some configuration. To add a new one, go to Workspaces $>$ Add Workspace and give it a name. This saves your current configuration as workspace. In the Organizer (at Workspaces $>$ Organize), you can see the respective key commands. You can also save the appearence of:
\begin{itemize}
	\item Transport panel
	\item Status line
	\item Info line
	\item Toolbars
	\item Inspector
\end{itemize}

In profiles, you can save:
\begin{itemize}
	\item Preferences
	\item Toolbar settings
	\item Global workspaces
	\item Track control settings
	\item Basically any kind of preset (except Control Room presets, track presets and plug-in presets)
	\item Key commands
\end{itemize}
So creating a profile of your setup is more convenient than applying all the steps in the \hyperlink{CustomSetup}{Custom Setup}. It also makes sense when more than one person with different preferences are working on the same computer. Profiles however don't contain Audio Connections or Project Templates. Profiles can be managed in Edit $>$ Profile Manager.

All those preference files are stored in AppData/Roaming/Steinberg/programname

\subsubsection{Lanes}

\begin{itemize}
	\item When activating solo of a lane, it simply mutes all the other lanes, so it can be heard in the project's context. Only one lane at a time can be solo.
	\item After having comped together a nice take, right click and Clean Up Lanes in order to remove empty ones
	\item Create suitable crossfades for overlaps and then Audio $>$ Advanced $>$ Delete Overlaps. Finally Audio $>$ Bounce Selection (or MIDI $>$ Bounce MIDI) to create a continuous event from the selected takes
	\item Comping Tool:
	\begin{itemize}
		\item Works on all Lanes simultaneously (e.g. for cutting begin, end or breaks in the middle)
		\item Click = Bring Take to front
		\item Shift + Click = Select
		\item Click + Drag = Creates new range for all takes and brings the one to front, on which the operation is performed
		\item Strg + Alt + Move = Adjust timing of selected takes
	\end{itemize}
\end{itemize}

\subsubsection{Track Versions}

Every track that can have track versions has a dropdown menu at its name. Most of the options are applied to all selected tracks.

Some usefull things:

\begin{itemize}
	\item Project $>$ Track Versions $>$ Select Tracks with Same Version ID.
	\item Project $>$ Track Versons $>$ Assign Common Version ID. The active version of the selected tracks get a common ID.
	\item Project $>$ Track Versions $>$ Duplicate Version: Creates new versions with the same contents as the active ones for all selected tracks
	\item Project $>$ Track Versions $>$ Rename Version works for all selected tracks
	\item Copy pasting between versions is possible by using the object selection or range tool and ctrl + c/v
\end{itemize}

\subsubsection{Jamming}

\begin{itemize}
	\item Use a chord track. It also can transpose notes in other tracks so they fit to the chord
\end{itemize}

\subsubsection{Logical Editor}\hypertarget{LogicalEditor}{}

Open via MIDI $>$ Logical Editor.

Super mighty, like a small MIDI programming tool inside Cubase. There are some useful presets, too (which could be used as a good starting point).

Upper half of the editor contains the conditions, lower half contains the orders to be executed.

Conditions can be created by drag\&dropping MIDI events.

All the stuff you program can be put into macros and keycommands.

It is usefull to know, how MIDI events are structured. A table can be found at p. 994

\subsubsection{Project Logical Editor}

Open via Project $>$ Project Logical Editor. Shitload of presets. Pretty much the same as the \hyperlink{LogicalEditor}{logical editor}, but bigger. E.g. you can search for media types. You can adjust length of audio events, changing whole tempo tracks, delete ALL panning automation, select stuff, bring all volumes down, change track names simultaneously, trigger macros, insert or delete stuff. And 

\subsubsection{Shortcuts}

\begin{itemize}
	\item Navigation
	\begin{itemize}
		\item numpad 1 / 2: Left / right \textbf{locator position}
		\item shift + number n: jump to \textbf{marker} n
	\end{itemize}
	\item Editing
	\begin{itemize}
		\item strg + d: Copies the selected part(s)
	\end{itemize}
	\item Replay \& Loop
	\begin{itemize}
		\item Select a part and press p, so the locators snap to the selection. Double-clicking a cycle marker does the same.
		\item ctrl/alt + click on ruler: Set left/right \textbf{locator}
		\item alt + shift + click in Project window: Sets the \textbf{cursor}
	\end{itemize}
\end{itemize}

\section{VST Plug-ins}

Setting things up is possible in Studio $>$ VST Plug-in Manager. Here you can also create collections for specific purposes (the + in the top right corner). Rescan is also possible, if you have just installed new plug-ins. The gear in the bottom left corner allows you managing plug-in paths.

For vst3 plug-ins, quick controls can be set directly from the plug-in through the context menu. There also is a nice Learn mode in the respective inspector tab. Just activate it, click on the QC you want to set and move the respective controller. The controller may also belong to a different track.

\subsection{Instruments}

In the overview, the context menu allows you to copy/paste settings and other stuff like loading and saving instrument/track presets. The displayed quick controls can also be connected to remote controllers in Studio $>$ Studio Setup $>$ VST Quick Controls. Just set the MIDI input, click Apply and check Learn. Then select on Control in the Control Name column and move the respective controller on your device. This remote controller setup is saved globally!

Instruments with side-chain can receive audio to
\begin{itemize}
	\item use the instrument as an effect plug-in
	\item use the audio as modulation source
	\item aplly ducking to the instrument
\end{itemize}

If too much CPU power is consumed, you can either freeze instruments (including insert effects) or suspend processing when no signals are received.

\subsubsection{Padshop 2}

Seems to be nice for atmospheric stuff, so actually pads. But hard synths also seem possible. And maybe it makes more sense to use this than the sampler track

\begin{itemize}
	\item Drag and drop samples
	\item Many posibilitiers to shape the sample, put envelopes etc.
\end{itemize}

\subsection{Effects}

\subsubsection{Delay}

\paragraph{Multi Tap Delay}

\begin{itemize}
	\item The dropdown in the top right corner gives you more options
	\begin{itemize}
		\item Wobbling
		\item Simulating a low quality delay by setting down the sample rate
		\item Damping, Low cut, high cut
	\end{itemize}
	\item Ducker: Good for vox $\rightarrow$ adjust it such that delay only takes effect when the singer is not singing
	\item Side chain support!
	\item In the bottom section: Can set up multiple effects for the delay like loop effects and stuff, all via drag and drop. To give the delay another color or cool effects and so on
	\item Setting taps (up to 8):
	\begin{itemize}
		\item By just double clicking in the window and drag and drop them (left right for time; up down for volume)
		\item Grid size can be set easily via drop down
		\item Tap rhythm: Activate it and then tap with the mouse where you want a tap while it plays
		\item In the panorama section, the taps' panning can be adjusted
		\item In the tap parameters section, their effect parameters can be adjusted (as well es in the tap effects section on the bottom) individually
		\item Top right knob can link the tabs, so you can move them all together
	\end{itemize}
\end{itemize}

\paragraph{PingPongDelay}

See it rather as a creative tool

\subsubsection{Distortion}

\paragraph{Distroyer}

Can be used for some nice sound shaping stuff. Make voices crispier, tracks brighter, drums bouncier.

\paragraph{Quadrafuzz}

Kind of a multiband distortion with delay. Could be cool for sound-shaping.

\paragraph{SoftClipper}

Second and Third are harmonics you can add to enrich the sound. Recommended for lead vocals to add some crunch or to dirty up acoustic guitars and brass.

\subsubsection{Dynamics}

\paragraph{MultibandCompressor}

Seems pretty powerful. You can set up the frequencies, set their volumes, define the compression, put specific bands to solo to hear what is going on there.

\paragraph{MultibandExpander}

"Anti-Compression" to give more dynamics in severall bands. Also has a side chain at the bottom, so one band can trigger another

\paragraph{VintageCompressor}

No threshold, nice level meters. Could be good for drums when using the punch. You can mix dry/wet in the compressor, which is cool. Can also be used in the channel strip.

\subsubsection{Filters}

\paragraph{Tonebooster}

Boosting frequency bands and adjusting the width. Can resemble a booster pedal for guitar sounds before going into an amp simulator.

\subsubsection{Mastering}

\paragraph{UV22HR}\hypertarget{UV-22HR}{}

Adds dithering. Important when mixing towards lower quality.

\subsubsection{External Effects} You need to set up the respective FX busses (sends and returns) in the Audio Connections (F4).

\subsubsection{Modulation}

\paragraph{Phaser}

\begin{itemize}
	\item Feedback: Defines the character
	\item Width: How much it moves between high and low frequencies
	\item Spatial: How much it moves between left and right
	\item LFO: Gives an automatic sweep between the existing frequencies
\end{itemize}

\paragraph{Studio Chorus} Has two stages. Can help simulating vintage pianos. Also nice for lead- and backing-vocals.

\paragraph{Tremolo} Nice for making Things sound vintage. Basically messes with the volume.

\paragraph{Vibrato} Has some nice presets (like doubler). Basically messes with the frequency.

\subsubsection{Pitch Shifter}

\paragraph{Octaver} Can be usefull to extract low rumbles and still make an instrument sound full (first apply  Octaver, then cut the low frequencies with an EQ).

\subsubsection{Spatial + Panner}

\paragraph{MonoToStereo} Well, it does what it says. It needs to be used on a stereo track. Seems to work nice on guitars, but sounds rather shitty on clean vox. But maybe suitable for backing vox.

But what about mono compatibility?

\subsection{MIDI Effects}

Can change or create new events or change properties like pitch. You can record the result to track by activating the respective button of the effect. When using a send effect, you can decide wether to apply it pre or post fader and inserts.

\section{Looks like a bug, but isn't}

\begin{itemize}
	\item Tracks that can only be added once are: Arranger, Chord, Signature, Tempo, Transpose, Video
	\item When auditioning, the signal is directly rooted to the control room. So if it isn't set up properly, you can't hear a thing.
\end{itemize}

\section{Scores}\label{Scores}

Progress: p. 18/210 (in manual for cubase10 $\rightarrow$ when proceeding here, we need to check if everything is still correct and where we need to continue)

\subsection{Basics}

\begin{itemize}
	\item Open score editor: Select parts $\rightarrow$ 'strg + r' for fullscreen. Alternatively you can choose 'Scores' as your editor in the lower portion of the project window
	\item Move project cursor: 'alt + shift + click'
	\item The left-side Editor Inispector has two modes if scores is opened in the bottom section
\end{itemize}

\subsection{Display Quantize}

Useful to display slightly off MIDI-notes in the way they should look like, without affecting the original recording. Tweaks are:

\begin{itemize}
	\item Smallest note length
	\item Smallest rests (eleminates rests which are smaller than this value, except when they are \textit{necessary}, e.g. at the beginning of a beat)
	\item If using different concepts (e.g. triols, quintols here and there), you should use the 'automatic dipslay quantize'
\end{itemize}

\subsection{Export}

\begin{itemize}
	\item Printing
	\begin{itemize}
		\item In the menu, choose 'Scores' $\rightarrow$ 'Page Mode'
		\item Printing parts without trailing empty bars: Preferences $>$ Scores $>$ Editing $>$ Unlock Layout When Editing Single Parts
	\end{itemize}
\end{itemize}


\section{Next important points}

\begin{itemize}
	\item Scores
	\item VariAudio (and how to tune it)
	\item PitchCorrect (and how to tune it)
	\item Create some nice makros and key commands
\end{itemize}

\end{document}